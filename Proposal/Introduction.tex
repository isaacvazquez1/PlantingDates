Las flores, frutas, vegetales y hongos representan un pilar fundamental en la cadena alimentaria \cite{Cassani_2022}. La industria agrícola, responsable de su producción continua, enfrenta desafíos crecientes debido a la incertidumbre respecto a las condiciones de cultivo, las cuales se han visto afectadas, en gran parte, por el cambio climático \cite{Parajuli_2019}.\\


Respecto a las estaciones del año, el cambio climático ha implicado un desfase en ellas. Las cuatro estaciones del año se definen a partir de criterios biológicos y físicos \cite{Trenberth_1983}. De esta manera, se ha conseguido determinar que, en el hemisferio norte, la duración del verano se ha prolongado significativamente; mientras que el resto de estaciones se han reducido y se especula que esta tendencia se exacerbe en los próximos años \cite{Wang_2021}, tanto por factores naturales como por factores ligados a la actividad humana \cite{Park_2018}.\\

Estas alteraciones repercuten en diversos aspectos, entre ellos el ciclo de producción agrícola, debido a variaciones inusuales e inesperadas en temperatura, precipitación y patrones de migración de polinizadores, entre otros factores. Los cuales, en conjunto, han modificado las zonas, temporadas y rendimiento de los cultivos \cite{Bayable_2021, Cuevas_2021, Parajuli_2019}.\\

Esto es especialmente relevante, ya que la estabilidad de los procesos productivos depende de cierta consistencia en las condiciones ambientales. Sin embargo, las condiciones actuales a las que nos enfrentamos han incrementado la inestabilidad en la seguridad alimentaria y el desabasto a la par de reducir la producción y exportación, impactando directamente la economía de las naciones \cite{Farooq_2022}.\\


Cada cultivo prospera dentro de umbrales ambientales específicos, que en muchos casos, han sido ya previamente descritos; e.g., el aguacate \cite{Moraes_2022, Rocha-Arroyo_2011}. En México, la producción del aguacate (\textit{Persea americana} Miller) se ha estandarizado para desarrollarse en dos etapas: crecimiento en vivero (10 -- 12 meses) y trasplante al suelo, donde permanecerá el resto de su vida productiva \cite{SAGARPA_2017}. La transición entre ambas, particularmente las cuatro semanas posteriores del cultivo en campo \cite{Premkumar_2002}, determina en gran medida el éxito del árbol a largo plazo.\\

De acuerdo con Campos--Rojas \textit{et al.} (2012) \cite{Campos-Rojas_2012}, la plantación en semilleros de viveros se recomienda que se ejecute entre marzo y mayo. Por tanto, al considerar la duración de la etapa de crecimiento en vivero antes mencionada, proveniente de \cite{SAGARPA_2017}, puede inferirse que el trasplante al suelo puede llevarse a cabo entre enero y mayo del año siguiente. Lo anterior es consistente con Salinas--Vargas \textit{et al.} (2021) \cite{CODESIN_2021} en donde se sugiere que, en Sinaloa, la plantación en campo se lleve a cabo entre los meses de febrero y marzo u octubre y noviembre, puesto que se aconseja evitar el trasplante durante el verano.\\


La transición de plantación en vivero hacia la plantación en suelo, conocida como impacto del trasplante, o \textit{transplant shock} en la literatura en inglés, implica un estrés biótico y abiótico significativo debido a la exposición a condiciones ambientales \cite{Close_2005, Mainhart_2024}. Por tanto, este fenómeno puede provocar la pérdida de plantas, afectando la producción, exportación y rentabilidad del cultivo \cite{Roka_2010}.\\

Tras el trasplante, la floración ocupa un papel crucial en la producción pues de ella depende la cantidad de frutos disponibles y, de acuerdo con Salazar--García \textit{et al.} (1999) \cite{Salazar_1999}, temporadas con temperaturas bajas y periodos nocturnos largos, seguidos de un pequeño incremento en la temperatura,  maximizan el desarrollo de las estructuras anatómicas que darán paso a la formación de flores de aguacate. En México, este proceso se ubicaría temporalmente entre el otoño y el invierno \cite{Acosta-Rangel_2021}.\\

Aunado a lo anterior, aún cuando la exposición a la luz solar resulta indispensable para la fotosíntesis de los árboles de aguacate, una exposición prolongada puede reducir la calidad de los frutos y generar quemaduras en los tallos y ramas, las cuales necrosan los tejidos y los dejan susceptibles a infecciones bacterianas y hongos \cite{Fischer_2022}. Asimismo, la exposición al sol interviene en el alza de la temperatura del suelo y la disminución de la humedad a través de la evaporación acelerada \cite{Haskell_2012}. Por tanto, se emplean mezclas de cal y cobre, en igualdad de proporciones, a manera de protector solar, además de cubrir las bases de los árboles con rastrojo para preservar la humedad y temperatura del suelo \cite{CODESIN_2021}.\\

De acuerdo con la Secretaría de Agricultura, Ganadería, Desarrollo Rural, Pesca y Alimentación (SAGARPA) \cite{SAGARPA_2017}, México es el principal productor y exportador mundial de aguacate actualmente, con 23 regiones productoras a nivel nacional. Para garantizar un desarrollo óptimo, se recomienda un pH del suelo entre 5.5 y 7, precipitaciones anuales de 1200 mm uniformemente distribuidos, ausencia de heladas y vientos calurosos secos, altitudes entre 800 y 2800 m.s.n.m. y temperaturas entre $12^\circ$ C y $33^\circ$ C \cite{CODESIN_2021}.\\


De esta manera, al vigilar y cuidar el proceso productivo, un árbol de aguacate que sobrevive al trasplante y se siembra en una zona y temporada ideal, será capaz de producir su primera producción comercial entre los tres y cuatro años de edad, con una producción promedio de 75 Kg, mas se estabilizará a entre 180 y 200 Kg de producción a los nueve años de edad \cite{Pena-Urquiza_2015}. Además, cada árbol tendrá aproximadamente una vida económicamente productiva de 40 años, la cual comienza a decaer un 5\% anual a partir de los 30 años de edad \cite{Mosquera_2015}.