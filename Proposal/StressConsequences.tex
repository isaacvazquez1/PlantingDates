A gran escala, la industria agrícola responde a una amplia variedad de elementos que, según su procedencia, permanencia, frecuencia e intensidad, pueden modificar su desarrollo de manera neutra, favorable o adversa. En este sentido, el estrés agrícola proviene de diversas fuentes y se manifiesta con distinta magnitud y recurrencia. En esta sección abordaremos particularmente las repercusiones del estrés agrícola atribuibles al cambio climático, clasificándolas en factores bióticos, abióticos y socioeconómicos.\\

El impacto del cambio climático ha sido profundamente significativo; en particular, la agricultura se considera la actividad más amenazada por este fenómeno \cite{Boyer_1982, Raza_2019}. Ello se debe a que, en general, la mayoría de estresores que inciden sobre los sistemas agrícolas tienen como origen directo o indirecto al cambio climático \cite{Olschewski_2024, Raza_2019, Surowka_2020}. Por lo anterior, centramos nuestro marco de trabajo en analizar las consecuencias del cambio climático sobre los cultivos de consumo, que se presentan a continuación.


\subsubsection{Factores bióticos}
\begin{itemize}[leftmargin=0cm, itemsep=0.5 cm]
	\item[]\textbf{Salud del cultivo\\}\\
	{Test\\ 
		
	Test1 }
	\item[]\textbf{Variabilidad genética}
\end{itemize}


\subsubsection{Factores abióticos}
\textbf{Cambio climático}
\textbf{Recursos hídricos}
\textbf{Recursos en suelo}

\subsubsection{Actividades humanas}
\textbf{Políticas públicas}
\textbf{Interés mercantil}
\textbf{Seguridad laboral en el campo}
