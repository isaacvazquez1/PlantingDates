A gran escala, la industria agrícola responde a una amplia variedad de elementos que, según su procedencia, permanencia, frecuencia e intensidad, pueden modificar su desarrollo de manera neutra, favorable o adversa. En este sentido, el estrés agrícola proviene de diversas fuentes y se manifiesta con distinta magnitud y recurrencia. En esta sección abordaremos particularmente las repercusiones del estrés agrícola atribuibles al cambio climático, clasificándolas en factores abióticos, bióticos y socioeconómicos.\\

El impacto del cambio climático ha sido profundamente significativo; en particular, la agricultura se considera la actividad más amenazada por este fenómeno \cite{Boyer_1982, Raza_2019}. Ello se debe a que, en general, la mayoría de estresores que inciden sobre los sistemas agrícolas tienen como origen directo o indirecto al cambio climático \cite{Olschewski_2024, Raza_2019, Surowka_2020}. Por lo anterior, centramos nuestro marco de trabajo en analizar las consecuencias del cambio climático sobre los cultivos de consumo, que se presentan a continuación.

\subsubsection{Factores abióticos}
\begin{itemize}[leftmargin=0cm, itemsep=0.5 cm]
\item[]\textbf{Daños térmicos\\}
\item[]\textbf{Daños hídricos\\}
\item[]\textbf{Daños solares\\}
\item[]\textbf{Daños mecánicos\\}
\item[]\textbf{Daños por gases atmosféricos\\}
\item[]\textbf{Daños edafológicos\\}
\item[]\textbf{Desplazamiento de climas\\}
\item[]\textbf{Desfase de estaciones y fenología\\}
\end{itemize}

\subsubsection{Factores bióticos}
\begin{itemize}[leftmargin=0cm, itemsep=0.5 cm]
	\item[]\textbf{Plagas por insectos\\}\\
	{Desde sus orígenes \cite{Mueller_2005}, la agricultura ha considerado la presencia de insectos, tanto para preservar y aprovechar a aquellos que resultan benéficos como para contener y evitar a los que se convierten en plaga \cite{Grogan_2014, Jones_2018}. La producción agrícola y las comunidades de insectos requieren mantener un equilibrio para prosperar mutuamente \cite{Mueller_2005}.\\ 
		
	Los cambios en la temperatura global han modificado las dinámicas poblacionales de los insectos, desde su distribución geográfica hasta su mortalidad \cite{Kiritani_2013}. Esto ha generado un desbalance entre cultivos huésped e insectos, además de alterar la relación con sus enemigos naturales \cite{Stange_2010, Thomson_2010}.\\
	
	La resiliencia de los ecosistemas puede mantener la estabilidad agrícola frente a perturbaciones ambientales ocasionadas por eventos extremos breves y esporádicos, como ondas de calor o sequías \cite{Asmamaw_2015}. Sin embargo, cuando estas condiciones se vuelven permanentes, la resiliencia se ve superada por lo que los ecosistemas transicionan hacia nuevos estados de manera irreversible \cite{Folke_2004}.\\
	
	El aumento sostenido de la temperatura ha favorecido la movilización de especies, lo que a su vez implica la introducción de insectos invasores \cite{Zhang_2025}. Lo anterior induce una competencia biológica en donde puede que los insectos invasores sean erradicados, erradiquen a los organismos locales o comiencen a cohabitar \cite{Ward_2007}.\\
	
	Si los nativos prevalecen, la estabilidad del ecosistema local se mantiene; en otro caso, la vegetación corre el riesgo de ser plagada por las especies introducidas. Por ejemplo, en Norteamérica, alrededor del 40\% de las plagas de insectos más relevantes corresponde a especies invasoras \cite{Niemela_1996}.\\
	
	Lo anterior se ha atribuido a que el aumento de las temperaturas y los cambios en la duración de las estaciones del año, de manera conjunta, han prolongado la duración de la época reproductiva, reducido la mortalidad por el invierno y adelantado la fecha cuando emergen los insectos en la primavera \cite{Kiritani_2013}. Por tanto, el número de generaciones anuales de insectos se ha visto favorecido \cite{Harvey_2022}, generando poblaciones cada vez más grandes.\\
	
	Del total de especies de insectos, solo el 3\% se consideran plagas de interés agrícola debido a su capacidad de colonización y rápida reproducción \cite{Garcia-Lara_2016}. No obstante, se proyecta que conforme aumenten las temperaturas también crezca el número de especies con relevancia agrícola, lo que implica mayores daños para la industria \cite{Deutsch_2018}.
	}
	
	\item[]\textbf{Debilitamiento de la polinización\\}\\
	{
	La polinización se refiere, de manera conjunta, a todos aquellos procesos de regulación que intervienen en la recolección, transferencia y deposición de granos de polen en las flores \cite{Liss_2013}. De igual manera, se contemplan aspectos como la frecuencia de tales eventos, los agentes que los llevan a cabo, los métodos a través de los cuales ocurren y la cobertura geográfica \cite{Carvalheiro_2010, Westerkamp_2000}.\\
	
	Los agentes polinizadores, en la naturaleza, incluyen a los insectos, las aves, mamíferos, el viento y el agua \cite{Amrutwad_2024}. La polinización zoógama, es decir, aquella en donde interviene la presencia e interacción de algún animal, destaca del resto pues se estima que más del 75\% de las flores en el mundo dependen de ella para mantener las poblaciones de plantas existentes \cite{Ollerton_2011}.\\
	
	La polinización anemófila, aquella que se lleva a cabo a través del viento, es considerada como la segunda fuente más importante dado que su aportación en la preservación de las plantas supera el 10\% \cite{Friedman_2009}. Más aún, respecto a cultivos para consumo, las anteriores formas se consideran como las únicas alternativas dado que la polinización hidrófila está presente en un reducido número de especies de plantas exclusivamente, no abundantes y sin interés en términos alimentarios para los humanos \cite{Clemente_2023}.\\
	
	Respecto a los polinizadores animales, el panorama es similar a lo discutido previamente referente a las plagas por insectos en tanto a la migración. En general, se ha observado la pérdida de hábitats de polinizadores \cite{Memmott_2007}, la cual ha implicado la migración y la redistribución geográfica de numerosas especies \cite{Kumar_2024, Sherwin_2012, Rafferty_2017}, derivado por la elevación en temperaturas, la disponibilidad de recursos y cambios en las relaciones bióticas.\\
	
	Adicionalmente, respecto a los insectos se ha reportado una reducción en su longevidad a consecuencia de alteraciones químicas en el néctar de las flores y en las concentraciones proteicas del polen \cite{Hoover_2012, Ziska_2016}. Tales cambios se desencadenan en ambientes con altas concentraciones de CO$_2$ y pocas precipitaciones \cite{Rafferty_2017}.\\
	
	Además, las sequías reducen el tamaño floral y la intensidad de colores \cite{Burkle_2016, Rafferty_2017}, volviéndolas menos visibles o llamativas para los polinizadores. De igual forma, los cultivos pueden verse afectados en aspectos genéticos puesto que la dispersión de semillas por parte de las aves se reduce \cite{Anderson_2016}.\\
	
	Respecto a los mamíferos, estos son responsables de polinizar plantas de difícil acceso para insectos y aves \cite{Carthew_1997}. De ellos destacan principalmente los murciélagos quienes llevan a cabo la polinización de plantas cuyas flores están disponibles en periodos sin luz solar, como el agave \cite{Trejo-Salazar_2016}.\\
	
	Por otra parte, la polinización anemófila es afectada directamente por cambios en la dirección, velocidad, humedad y temperatura del viento \cite{Cresswell_2010}. En consecuencia, tales condiciones modifican la humedad en el polen, reduciendo su viabilidad cuando está falto de humedad y dificultando su motilidad al estar húmedo en exceso \cite{Fonseca_2005}.\\
	
	Adicionalmente, si el viento transporta al polen por más de dos horas, este puede dejarlo inviable por el secado o por el aislamiento causado por haber sido depositado fuera de un área de cultivo \cite{Luna_2001}. El caso de la polinización hidrófila resulta similar, dado que las corrientes pueden reducir la fertilidad y depositarlo en sitios sin relevancia reproductiva \cite{Ackerman_2000}.\\
	
	La polinización tiene un papel crucial en la supervivencia, prevalencia y variabilidad genética en las plantas \cite{Kearns_1998}. Más aún, para fines de consumo, es también fundamental para la cosecha de productos y semillas de buena calidad, tanto en sabor como en tamaño \cite{Amrutwad_2024}, lo cual repercute más notablemente en los cultivos perennes por el tiempo que se mantienen en el campo \cite{Rafferty_2017}.\\ 
	
}
	
	\item[]\textbf{Salud de los cultivos\\}\\
	Similar a lo que ocurre en los seres humanos, los cultivos se encuentran expuestos a diversos organismos patógenos que pueden comprometer su salud \cite{Savary_2020}. Estas enfermedades pueden adquirirse tanto a través de los órganos subterráneos \cite{Katan_2017} como en los órganos que yacen al aire libre \cite{Aylor_2003}, y de acuerdo con su patrón epidémico y sus consecuencias se clasifican en crónicas, agudas o emergentes \cite{Savary_2017}.\\
	
	Las enfermedades crónicas son aquellas que ocurren regularmente en cada temporada y afectan amplias extensiones de cultivo \cite{Zadoks_1979}. En contraste, las enfermedades agudas aparecen de manera irregular tanto en el tiempo como en el espacio \cite{Savary_2020, Zadoks_1979}. Finalmente, las enfermedades emergentes son aquellas que se encuentran en proceso de expansión, ya sea en su distribución geográfica o en la gama de especies que logran infectar \cite{Anderson_2004}.\\
	
	Los agentes responsables de fitoenfermedades incluyen a los hongos, virus, bacterias y nematodos  \cite{Savary_2017B}. El tamaño y localización de sus poblaciones se ha modificado en consecuencia del cambio climático generando una tendencia a que más enfermedades sean consideradas como emergentes, el reporte de nuevas enfermedades y mayor virulencia \cite{Lahlali_2024}.\\
	
	Lo anterior no solo repercute directamente en la agricultura, sino que pone en riesgo la integridad ecológica del ecosistema, en general \cite{Lahlali_2024}. Por tanto, la estabilidad de los microbiomas del suelo y la superficie se ve comprometida rápidamente, afectando su prevalencia a largo plazo \cite{Jansson_2020}, al reducir la diversidad microbiana, alterar la descomposición de materia orgánica y la retención de nutrientes \cite{Iqbal_2025}.\\	
	
	De esta manera, los daños ocasionados por patógenos, como la aceleración del secado de las hojas, la reducción de la fotosíntesis o la pérdida de firmeza de los tejidos vegetales, se presentan con mayor frecuencia \cite{Savary_2020}. Además, la simultánea disminución de los microorganismos benéficos, que ciclan nutrientes y mitigan patógenos, favorece un desequilibrio agrícola que pone en riesgo la salud de los cultivos \cite{Jansson_2020}.
	
	\item[]\textbf{Competencia entre plantas\\}\\
	
	En las secciones previas hemos explorado las relaciones biológicas entre los cultivos y diversos organismos; sin embargo, el cambio climático también ha transformado las interacciones que ocurren entre las propias especies vegetales \cite{Guo_2025}. En esta sección nos enfocaremos en analizar las dinámicas de competencia que los cultivos establecen con la maleza, las plantas parasitarias, los cultivos introducidos y aquellos generados mediante organismos genéticamente modificados.\\
	
	En términos generales, la maleza agrupa a todas aquellas plantas en un sistema agrícola que crecen de manera no deseada dado que poseen mecanismos de dispersión adaptables y eficientes \cite{Rana_2016}. A nivel nutricional, la maleza se comporta de manera similar a los cultivos; no obstante, por cuestiones genéticas, su agresividad y capacidad de expansión biogeográfica pueden verse favorecidas por el alza en las concentraciones de CO$_2$, las altas temperaturas y el estrés hídrico \cite{Upasani_2018}.\\
	
	Lo anterior implica que la maleza, como competencia para los cultivos, posee habilidades de adaptación y una fuerte capacidad de capturar los recursos disponibles \cite{Rana_2016}. En contraste, respecto a las plantas parasitarias también se ha reportado una ampliación tanto geográfica como en la gama de especies hospederas \cite{Zamora_2019}.\\
	
	En el caso de las plantas parásitas, estas no compiten directamente por los recursos, sino que los extraen de manera indirecta de la planta hospedera \cite{Watson_2022}. Lo anterior provoca un reajuste en la distribución de la energía reservada por la planta, lo que conlleva a un aumento de la transpiración en el hospedero, frecuentemente asociado a un enfriamiento de sus tejidos \cite{Pincebourde_2019}.\\
	
	Este fenómeno reduce la producción de flores y frutos \cite{Watson_2022}, pero puede implicar consecuencias fatales. En zonas donde, a consecuencia del cambio climático, los recursos hídricos han incrementado, este efecto genera condiciones de alta humedad que favorecen el desarrollo de co-infecciones por patógenos \cite{Velasquez_2018}. Mientras que en zonas con escasez de agua, el incremento en la transpiración puede resultar mortal tanto para la planta parásita como para el hospedero \cite{Watson_2022}.\\
	
	Por tanto, la presencia tanto de maleza como de plantas parasitarias es indeseable, desde un punto de vista económico y productivo. Desde esta perspectiva, también destaca la introducción de cultivos no nativos y aquellos generados mediante organismos genéticamente modificados, los cuales se abordan a continuación.\\
	
	En un sistema agrícola, los cultivos no nativos pueden ser introducidos de manera natural, al aprovechar oportunidades de colonización; de manera artificial, por decisión humana; o mediante una combinación de ambas estrategias \cite{Webber_2012}. En cualquiera de los casos, esto afecta directamente a la biodiversidad local y la sinergia del ecosistema \cite{Sorte_2013}.\\
	
	La introducción de especies ocurre principalmente por los cambios en el ambiente \cite{Byers_2002}. En cuanto a la introducción natural, esta ocurre debido a que las especies han hallado condiciones más favorables para su desarrollo \cite{Sorte_2013}. Mientras que la introducción artificial se lleva a cabo, principalmente, por cuestiones productivas y su rentabilidad comercial \cite{Hodge_1956}.\\
	
	En tierras de cultivo, mayoritariamente, la introducción de especies se realiza de manera artificial \cite{Steen_2024}; mientras que en cultivos acuáticos, la introducción de especies vegetales se realiza principalmente de manera natural \cite{Sorte_2013}. No obstante, la sustitución de cultivos nativos por especies no nativas ha ido en aumento; e.g., en Europa se estima que el 40\% de las especies nativas serán sustituidas para el año 2080 \cite{Thuiller_2005}.\\
	
	A diferencia de los cultivos no nativos, cuya introducción puede ser natural o artificial, los cultivos genéticamente modificados dependen exclusivamente de la intervención humana. En este caso, las compañías fabricantes diseñan y editan las semillas con el fin de obtener cultivos resistentes a condiciones ambientales adversas, plagas y enfermedades \cite{Verma_2011} a la par de asegurar una alta producción de calidad nutrimental y comercial \cite{Cheng_2024}.\\
	
	
	Por otra parte, al consumir los productos resultantes la edición genética, potencialmente pueden presentarse nuevas reacciones alérgicas impredecibles y alteraciones en los sistemas digestivo e inmunológico \cite{Verma_2011}. Adicionalmente, este tipo de cultivos suele incorporar tecnologías de restricción de uso genético, coloquialmente conocidas como \textit{genes suicidas}, que hacen infértiles a las semillas que se obtienen, por lo que estos cultivos tienen un único ciclo productivo \cite{Daniell_2002}.\\
	
	
	Lo anterior se realiza como una medida de bioseguridad; no obstante tiene repercusiones económicas adversas para los productores \cite{VanAcker_2007}. De igual forma, al adoptar este tipo de cultivos, las especies nativas y, en especial, aquellas endémicas se ven altamente amenazadas corriendo, incluso, el riesgo de la extinción \cite{Kumar_2020}.\\
	
	
	En gran medida, la desventaja que presentan los cultivos frente a otras especies vegetales puede atribuirse al refinamiento genético al que han sido sometidos durante el proceso de domesticación \cite{Moreira_2018}. Este proceso busca obtener características productivas deseables, pero ha reducido significativamente la agresividad y capacidad competitiva de los cultivos frente a otras plantas \cite{DaSilva_2007, Concenco_2012}.
	
\end{itemize}




\subsubsection{Factores socioeconómicos}
\begin{itemize}[leftmargin=0cm, itemsep=0.5 cm]
\item[]\textbf{Superficies cultivables\\}
\item[]\textbf{Escasez laboral\\}
\item[]\textbf{Volatilidad productiva\\}
\item[]\textbf{Toxicidad por agroquímicos\\}
\end{itemize}