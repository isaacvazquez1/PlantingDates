A gran escala, la industria agrícola responde a una amplia variedad de elementos que, según su procedencia, permanencia, frecuencia e intensidad, pueden modificar su desarrollo de manera neutra, favorable o adversa. En este sentido, el estrés agrícola proviene de diversas fuentes y se manifiesta con distinta magnitud y recurrencia. En esta sección abordaremos particularmente las repercusiones del estrés agrícola atribuibles al cambio climático, clasificándolas en factores abióticos, bióticos y socioeconómicos.\\

El impacto del cambio climático ha sido profundamente significativo; en particular, la agricultura se considera la actividad más amenazada por este fenómeno \cite{Boyer_1982, Raza_2019}. Ello se debe a que, en general, la mayoría de estresores que inciden sobre los sistemas agrícolas tienen como origen directo o indirecto al cambio climático \cite{Olschewski_2024, Raza_2019, Surowka_2020}. Por lo anterior, centramos nuestro marco de trabajo en analizar las consecuencias del cambio climático sobre los cultivos de consumo, que se presentan a continuación.

\subsubsection{Factores abióticos}
\begin{itemize}[leftmargin=0cm, itemsep=0.5 cm]
\item[]\textbf{Estrés climático\\}
\item[]\textbf{Estrés agroambiental\\}
\item[]\textbf{Desplazamiento de climas\\}
\item[]\textbf{Desfase de estaciones y fenología\\}
\end{itemize}

\subsubsection{Factores bióticos}
\begin{itemize}[leftmargin=0cm, itemsep=0.5 cm]
	\item[]\textbf{Plagas por insectos\\}\\
	{Desde sus orígenes \cite{Mueller_2005}, la agricultura ha considerado la presencia de insectos, tanto para preservar y aprovechar a aquellos que resultan benéficos como para contener y evitar a los que se convierten en plaga \cite{Grogan_2014, Jones_2018}. La producción agrícola y las comunidades de insectos requieren mantener un equilibrio para prosperar mutuamente \cite{Mueller_2005}.\\ 
		
	Los cambios en la temperatura global han modificado las dinámicas poblacionales de los insectos, desde su distribución geográfica hasta su mortalidad \cite{Kiritani_2013}. Esto ha generado un desbalance entre cultivos huésped e insectos, además de alterar la relación con sus enemigos naturales \cite{Stange_2010, Thomson_2010}.\\
	
	La resiliencia de los ecosistemas puede mantener la estabilidad agrícola frente a perturbaciones ambientales ocasionadas por eventos extremos breves y esporádicos, como ondas de calor o sequías \cite{Asmamaw_2015}. Sin embargo, cuando estas condiciones se vuelven permanentes, la resiliencia se ve superada por lo que los ecosistemas transicionan hacia nuevos estados de manera irreversible \cite{Folke_2004}.\\
	
	El aumento sostenido de la temperatura ha favorecido la movilización de especies, lo que a su vez implica la introducción de insectos invasores \cite{Zhang_2025}. Lo anterior induce una competencia biológica en donde puede que los insectos invasores sean erradicados, erradiquen a los organismos locales o comiencen a cohabitar \cite{Ward_2007}.\\
	
	Si los nativos prevalecen, la estabilidad del ecosistema local se mantiene; en otro caso, la vegetación corre el riesgo de ser plagada por las especies introducidas. Por ejemplo, en Norteamérica, alrededor del 40\% de las plagas de insectos más relevantes corresponde a especies invasoras \cite{Niemela_1996}.\\
	
	
	Lo anterior se ha atribuido a que el aumento de las temperaturas y los cambios en la duración de las estaciones del año, de manera conjunta, han prolongado la duración de la época reproductiva, reducido la mortalidad por el invierno y adelantado la fecha cuando emergen los insectos en la primavera \cite{Kiritani_2013}. Por tanto, el número de generaciones anuales de insectos se ha visto favorecido \cite{Harvey_2022}, generando poblaciones cada vez más grandes.\\
	
	Del total de especies de insectos, solo el 3\% se consideran plagas de interés agrícola debido a su capacidad de colonización y rápida reproducción \cite{Garcia-Lara_2016}. No obstante, se proyecta que conforme aumenten las temperaturas también crezca el número de especies con relevancia agrícola, lo que implica mayores daños para la industria \cite{Deutsch_2018}.
	}
	\item[]\textbf{Deficiencias en la polinización\\}\\
	{La polinización se refiere, de manera conjunta, a todos aquellos procesos de regulación que intervienen en la recolección, transferencia y deposición de granos de polen en las flores \cite{Liss_2013}. De igual manera, se contemplan aspectos como la frecuencia de tales eventos, los agentes que los llevan a cabo, los métodos a través de los cuales ocurren y la cobertura de geográfica \cite{Carvalheiro_2010, Westerkamp_2000}.\\
	
	Los agentes polinizadores, en la naturaleza, incluyen a los insectos, las aves, mamíferos, el viento y el agua \cite{Amrutwad_2024}. La polinización zoógama, es decir, aquella en donde interviene la presencia e interacción de algún animal, destaca del resto pues se estima que más del 75\% de las flores en el mundo dependen de ella para mantener las poblaciones de plantas existentes \cite{Ollerton_2011}.\\
	
	La polinización adenógama, aquella que se lleva a cabo a través del viento, es considerada como la segunda fuente más importante dado que su aportación en la preservación de las plantas supera el 10\% \cite{Friedman_2009}. Más aún, respecto a cultivos para consumo, las anteriores formas se consideran como las únicas alternativas dado que la polinización hidrógama está presente en un reducido número de especies de plantas exclusivamente, no abundantes y sin interés en términos alimentarios para los humanos \cite{Clemente_2023}.\\
	
	Respecto a los polinizadores animales, el panorama es similar a lo discutido previamente referente a las plagas por insectos en tanto a la migración. En general, se ha observado la pérdida de hábitats de polinizadores \cite{Memmott_2007}, la cual ha implicado la migración y la redistribución geográfica de numerosas especies \cite{Kumar_2024, Sherwin_2012,  Rafferty_2017}, derivado por la elevación en temperaturas, la disponibilidad de recursos y cambios en la relaciones bióticas.\\
	
	Adicionalmente, respecto a los insectos se ha reportado una reducción en su longevidad a consecuencia de alteraciones químicas en el néctar de las flores y las concentraciones protéicas en el polen \cite{Hoover_2012, Ziska_2016}. Tales cambios se desencadenan en ambientes con altas concentraciones de CO$_2$ y pocas precipitaciones \cite{Rafferty_2017}.\\
	
	Además, las sequías reducen el tamaño floral e intensidad de colores \cite{Burkle_2016, Rafferty_2017}, volviéndolas menos visibles o llamativas. De igual forma, los cultivos pueden verse afectados en aspectos genéticos puesto a que la dispersión de semillas por parte de las aves se reduce \cite{Anderson_2016}.\\
	
	Respecto a los mamíferos, destacan principalmente los murciélagos quienes, junto con otros animales nocturnos, llevan a cabo la polinización de plantas cuyas flores están disponibles en periodos sin luz solar, como el agave \cite{Trejo-Salazar_2016}.
	}
	
	\item[]\textbf{Salud de los cultivos\\}
	\item[]\textbf{Competencia biológica\\}
\end{itemize}




\subsubsection{Factores socioeconómicos}
\begin{itemize}[leftmargin=0cm, itemsep=0.5 cm]
\item[]\textbf{Superficies cultivables\\}
\item[]\textbf{Escasez laboral\\}
\item[]\textbf{Volatilidad productiva\\}
\item[]\textbf{Toxicidad por agroquímicos\\}
\end{itemize}