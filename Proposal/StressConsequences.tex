A gran escala, la industria agrícola responde a una amplia variedad de elementos que, según su procedencia, permanencia, frecuencia e intensidad, pueden modificar su desarrollo de manera neutra, favorable o adversa. En este sentido, el estrés agrícola proviene de diversas fuentes y se manifiesta con distinta magnitud y recurrencia. En esta sección abordaremos particularmente las repercusiones del estrés agrícola atribuibles al cambio climático, clasificándolas, por su origen, en factores abióticos, bióticos y socioeconómicos.\\

El impacto del cambio climático ha sido profundamente significativo; en particular, la agricultura se considera la actividad más amenazada por este fenómeno \cite{Boyer_1982, Raza_2019}. Ello se debe a que, en general, la mayoría de estresores que inciden sobre los sistemas agrícolas tienen como origen directo o indirecto al cambio climático \cite{Olschewski_2024, Raza_2019, Surowka_2020}. Por lo anterior, centramos nuestro marco de trabajo en analizar las consecuencias del cambio climático sobre los cultivos de consumo, que se presentan a continuación.

\subsubsection{Factores abióticos\\}

\begin{itemize}[leftmargin=0cm, itemsep=0.5 cm]
\item[]\textbf{Daños térmicos\\}\\

Uno de los efectos más relacionados con el cambio climático es la variación de temperaturas, siendo más frecuente el aumento en las temperaturas; no obstante, también influyen los eventos asociados a bajas temperaturas \cite{Cohen_2021}. En ambos casos, tales cambios térmicos tienen implicaciones en el ecosistema; particularmente sobre los cultivos, mismas que discutimos en esta sección.\\

El alza en la temperatura promedio del planeta comenzó a documentarse entre las décadas de 1970 y 1980 \cite{Easterling_2024}. Desde dicho momento, empleando distintos métodos, se ha estimado el incremento que sucederá gradualmente hasta el año 2100, reportando un aumento de entre 3 $^\circ$C, en escenarios optimistas, y de hasta 6 $^\circ$C, en los escenarios más pesimistas \cite{Hroub_2025}.\\

Esto resulta de sumo interés dado que los cultivos, al exponerse sistemáticamente a temperaturas que superan su umbral fenológico, pueden sufrir daños permanentes sobre su crecimiento y desarrollo \cite{Hasanuzzaman_2013}. Durante la etapa de crecimiento reproductivo, las altas temperaturas impiden que los granos de polen se hinchen, reduciendo así su liberación \cite{Ngoune_2020}.\\

En consecuencia, la viabilidad y consistencia del polen, al igual que las poblaciones de polinizadores, se reducen \cite{Antle_1993}. Además, el calor como estresor afecta negativamente la formación y funcionamiento de órganos vegetativos y reproductivos, lo cual resulta de particular interés respecto al desarrollo de flores y hojas \cite{Ngoune_2020}.\\

De esta manera, tales alteraciones funcionales interfieren tanto con el proceso de reproducción como con la eficiencia de la fotosíntesis al alterar los órganos sexuales y foliares, respectivamente \cite{Kumaraswamy_2016, Ngoune_2020}. Asimismo, las altas temperaturas reducen el porcentaje de germinación en las semillas y aceleran los ciclos fenológicos, adelantando las épocas de floración, cosecha y maduración \cite{Antle_1993}.\\

Como mencionamos, el incremento térmico es el fenómeno más documentado, pero las bajas temperaturas también repercuten en los cultivos. Estas pueden afectar el desarrollo de las plantas en aspectos como un letargo en el crecimiento y un bajo conteo de hojas \cite{Ngoune_2020}. De igual forma, tienen efectos adversos que ocasionan una mala pigmentación en las hojas, pérdida de plántulas y bajos índices de germinación \cite{Zhang_2020}.\\

El estrés térmico ha estado presente en la agricultura mas se ha intensificado por el cambio climático. Por lo tanto, las consecuencias sobre los cultivos se han ampliado y evidenciado no solo a través de pérdidas a nivel de producción, sino también a nivel de organismos vivos.





%heat stress with 3 ◦C global warming could reduce labor capacity in agriculture by 30%–50%, increasing food prices and requiring much higher levels of employment in the farm sector.

%Heat stress on agricultural workers exacerbates crop impacts of climate change
%Cicero Z de Lima




\item[]\textbf{Daños hídricos\\}\\

La industria agrícola hace uso de grandes cantidades de agua para sostenerse y cubrir la demanda comercial \cite{Li_2022}. Por ello, la disponibilidad de agua desempeña un rol crucial para los cultivos que dependen exclusivamente de fuentes naturales y en aquellos que integran sistemas de riego \cite{Rockstrom_2010}.\\

Tanto los cultivos de riego como los de temporal están expuestos al abastecimiento natural, como las lluvias y las reservas subterráneas \cite{Rockstrom_2010}. En consecuencia, ante un déficit hídrico, los cultivos de temporal resultan más afectados \cite{Chen_2025}, mientras que ante la sobresaturación de agua, ambos tipos de cultivo pueden sufrir daños \cite{VanderMolen_2007}.\\

Los desabastos de agua, o sequías, se presentan cuando la cantidad de agua disponible resulta insuficiente para reponer la humedad perdida por el proceso de evapotranspiración \cite{Kumaraswamy_2016}. En tal proceso  tanto el ambiente como los cultivos pierden hidratación por la evaporación y por la transpiración vegetal, respectivamente \cite{Katul_2012}.\\

Esta deficiencia hídrica limita el desarrollo, el crecimiento y la producción de las plantas, llegando potencialmente a ocasionar su muerte \cite{Ngoune_2020}. Dichos efectos se han atribuido a la acumulación de ácido abscísico en las células vegetales, una hormona vegetal que induce el cierre de los estomas \cite{Li_2014}.\\

Lo anterior conlleva a la desnutrición de la planta dado que se impide la entrada de CO$_2$ y, por tanto, la fotosíntesis se ve reducida \cite{Gahir_2021}. Asimismo, la calidad de productos y semillas se ve afectada cuando la planta prioriza la búsqueda de agua mediante la expansión de su sistema radicular \cite{Comas_2013}.\\

La ausencia de agua está más estrechamente asociada con causas naturales \cite{AghaKouchak_2021}; en contraste, el exceso de agua puede ser consecuencia tanto de eventos naturales como de intervenciones humanas \cite{Patra_2024}. En los cultivos, el exceso hídrico puede originarse por factores como la variabilidad climática o por la topografía, pero también por aspectos como la compactación del suelo debido al uso de maquinaria pesada o el riego excesivo \cite{Patra_2024}.\\

Independiente a si el cultivo está sumergido en su totalidad o si únicamente sus raíces se encuentran bajo el agua, esto impone condiciones anaeróbicas en el organismo \cite{Striker_2012}. El ambiente restrictivo interrumpe el intercambio de gases y provoca la acumulación de CO$_2$ y etileno, junto con la falta de oxígeno \cite{Ngoune_2020}.\\

En consecuencia, la fotosíntesis se torna ineficiente y la energía disminuye, afectando con mayor rapidez a aquellos tejidos sumergidos \cite{Vartapetian_1997}. Esto conduce a un crecimiento limitado generalizado así como al debilitamiento y fractura de raíces, lo que puede provocar el desprendimiento o incluso la muerte la planta \cite{Patra_2024}.\\

La dependencia de la agricultura al agua es tan masiva que constituye el sector económico con la mayor demanda de este recurso \cite{Rockstrom_2010}. Por tanto, los cultivos son altamente susceptibles tanto a la escasez como al exceso de agua; particularmente aquellos cultivos de temporal, cuyo mantenimiento representa el $77.46\%$ del agua consumida por la agricultura a nivel mundial anualmente \cite{deFraiture_2007}.



%Sequía induce un mayor estrés hidrico implica más ¿presión? por parte de los sistemas de riego

\item[]\textbf{Daños solares\\}\\
\item[]\textbf{Daños mecánicos\\}\\
\item[]\textbf{Daños por gases atmosféricos\\}\\
\item[]\textbf{Daños edafológicos\\}\\
El suelo es más que solo un lugar en donde los cultivos permanecen la mayor parte de su vida económicamente productiva \cite{Passioura_2002}. La fijación al terreno, la obtención de nutrientes, la disponibilidad a reservas de agua y el acceso a microorganismos benéficos son aspectos proporcionados por el suelo \cite{Hatfield_2017}.\\

La importancia del suelo es tal que se estima que, directa e indirectamente, más del 95\% de la comida consumida por los humanos es provista por el suelo \cite{Kopittke_2019}. No obstante, el cambio climático ha implicado la acidificación, la salinización y el mal funcionamiento del ciclo de reabastecimiento de nutrientes \cite{Hamidov_2018}.\\

Consecuencias cuyo impacto sobre los cultivos discutimos a continuación. En gran medida, la fertilidad del suelo está en función de su disponibilidad de micronutrientes; como el hierro o el cobre, y de macronutrientes; por ejemplo, el calcio y el potasio \cite{Berbecea_2024}. Este factor es crucial para cada etapa del desarrollo de cultivos, especialmente, respecto a la producción \cite{Wang_2014}.\\

Por una parte, la falta de tales nutrientes causa deficiencias en las plantas cuya gama comprende desde efectos sobre la coloración hasta la necrosis de tejidos \cite{Uchida_2000}. A nivel de producción destacan, por ejemplo, el raquistimo, baja en el contenido proteico en semillas, reducción de flores y cosechas, cambios en el tiempo de maduración \cite{Uchida_2000}.\\

Por otra parte, el exceso de tales nutrientes conlleva a toxicidad en los cultivos \cite{Ngoune_2020}. Las toxicidades más importantes están relacionadas con el manganeso, el boro y el alumnio; no obstante, las opciones para su control son muy limitadas \cite{Langridge_2022}.\\ 

Las toxicidades causadas por estos tres elementos incluyen cambios en la coloración de hojas, necrosis en áreas puntuales, bloqueo de crecimiento de raíces y dificultad para el desarrollo del organismo \cite{Uchida_2000}. En la ingesta, las deficiencias se traducen en una mala nutrición; mientras que las toxicidades representan un riesgo para humanos y animales \cite{McLaughlin_1999}.\\

Los suelos también enfrentan un problema de salinidad, en donde la concentración de sales se acumula a través del riego y/o las temporadas de sequía \cite{Atta_2023}. La concentración de estas sales interviene en la absorción de nutrientes y agua; así, una alta acumulación impide dicha absorción \cite{Ngoune_2020}.\\

Más aún, el exceso de salinidad facilita la entrada de iones tóxicos que afectan a las membranas celulares y a la ejecución de actividades metabólicas \cite{Kumar_2014}. Consecuentemente, disminuye la producción y se reduce el desarrollo de raíces y órganos; simultáneamente, aumenta el número de moléculas de oxígeno inestable, las cuales interfieren con la energía del organismo \cite{Zorb_2019}.\\

Por último, la acidificación del suelo se atribuye a la acción conjunta de altas temperaturas aunado con grandes concentraciones de CO$_2$ \cite{Oishy_2025}. De esta manera, a través de lluvias y sequías, ácidos orgánicos se infiltran al suelo, lo que en consecuencia favorece la actividad microbiana que acelera la descomposición de materia orgánica\cite{Ferdush_2023}.\\

Además, esta reducción en el pH del suelo vuelve inadecuados a los terrenos dado que los cultivos requieren de determinados niveles de acidez para mantenerse con vida \cite{Ngoune_2020}. De esta manera, las condiciones del suelo moldean a los cultivos que albergan, desde su nivel y calidad de producción hasta su supervivencia y bienestar.

  




\item[]\textbf{Desplazamiento de climas\\}\\
\item[]\textbf{Desfase de estaciones y fenología\\}\\
\end{itemize}



\begin{table}[ht]
	\centering
	\caption{Consecuencias de origen abiótico del cambio climático sobre los cultivos}
	\begin{tabularx}{\textwidth}{|l|X|}
		\hline
		\textbf{Consecuencia sobre el cultivo} & \textbf{Descripción resumida / Factores causales} \\ \hline
		Daños térmicos & Estrés por calor. \linebreak%
		 Cambios en su umbral fenológico (maduración temprana, prolongamiento de la fase reproductiva, menor porcentaje de germinación).\linebreak%
		 Menor viabilidad y consistencia del polen.\linebreak%
		 Liberación reducida del polen a falta de su hinchamiento.\linebreak%
		 Ineficiencia en la formación y funcionamiento de órganos.\linebreak%
		 Estrés por frío. \linebreak%
		 Baja en la tasa de crecimiento.\linebreak%
		 Reducción en el número de hojas.\linebreak%
		 Pérdida de plántulas.\linebreak%
		 Bajo índice de germinación\linebreak%
		 \\ \hline
		Daños hídricos & Sequías.\linebreak% 
		Limitación del desarrollo, crecimiento y producción.\linebreak%
		Potencial muerte.\linebreak%
		Acumulación de CO$_2$ por el cierre de estomas, reduciendo la fotosíntesis.\linebreak%
		Prioriza la expansión de raíces para la búsqueda de agua, por lo que la calidad de frutos y semillas baja.\linebreak%
		Inundaciones.\linebreak% 
		Crea ambientes anaeróbicos que impide el intercambio de gases. \linebreak%
		Falta de oxígeno y CO$_2$ acumulado.\linebreak%
		Reducción de energía y de fotosíntesis debilitando a la planta .\linebreak%
		Raíces quebradizas que comprometen la fijación de la planta al suelo.\linebreak%
		\\ \hline
		Daños solares & Exceso de radiación y radiación UV provoca quemaduras foliares, fotoinhibición y reducción de fotosíntesis. \\ \hline
		Daños mecánicos & Viento rompe tallos, flores y hojas; aumenta evapotranspiración y erosión del suelo. \\ \hline
		Daños por gases atmosféricos & Elevación de CO$_2$, O$_3$ y contaminantes: algunos incrementan biomasa (CO$_2$) pero reducen calidad nutricional; O$_3$ provoca necrosis y pérdida de rendimiento. \\ \hline
		Daños edafológicos & Degradación, salinización, acidificación y cambios en humedad reducen disponibilidad de nutrientes y agua. \\ \hline
		Desplazamiento de climas & Corrimiento de zonas aptas para cultivos hacia nuevas latitudes/altitudes debido a cambios de temperatura y precipitación. \\ \hline
		Desfase de estaciones y fenología & Alteraciones en duración de estaciones → desincronización del calendario agrícola, floración y maduración. \\ \hline
	\end{tabularx}
\end{table}

\subsubsection{Factores bióticos}
\begin{itemize}[leftmargin=0cm, itemsep=0.5 cm]
	\item[]\textbf{Plagas por insectos\\}\\
	{Desde sus orígenes \cite{Mueller_2005}, la agricultura ha considerado la presencia de insectos, tanto para preservar y aprovechar a aquellos que resultan benéficos como para contener y evitar a los que se convierten en plaga \cite{Grogan_2014, Jones_2018}. La producción agrícola y las comunidades de insectos requieren mantener un equilibrio para prosperar mutuamente \cite{Mueller_2005}.\\ 
		
	Los cambios en la temperatura global han modificado las dinámicas poblacionales de los insectos, desde su distribución geográfica hasta su mortalidad \cite{Kiritani_2013}. Esto ha generado un desbalance entre cultivos huésped e insectos, además de alterar la relación con sus enemigos naturales \cite{Stange_2010, Thomson_2010}.\\
	
	La resiliencia de los ecosistemas puede mantener la estabilidad agrícola frente a perturbaciones ambientales ocasionadas por eventos extremos breves y esporádicos, como ondas de calor o sequías \cite{Asmamaw_2015}. Sin embargo, cuando estas condiciones se vuelven permanentes, la resiliencia se ve superada por lo que los ecosistemas transicionan hacia nuevos estados de manera irreversible \cite{Folke_2004}.\\
	
	El aumento sostenido de la temperatura ha favorecido la movilización de especies, lo que a su vez implica la introducción de insectos invasores \cite{Zhang_2025}. Lo anterior induce una competencia biológica en donde puede que los insectos invasores sean erradicados, erradiquen a los organismos locales o comiencen a cohabitar \cite{Ward_2007}.\\
	
	Si los nativos prevalecen, la estabilidad del ecosistema local se mantiene; en otro caso, la vegetación corre el riesgo de ser plagada por las especies introducidas. Por ejemplo, en Norteamérica, alrededor del 40\% de las plagas de insectos más relevantes corresponde a especies invasoras \cite{Niemela_1996}.\\
	
	Lo anterior se ha atribuido a que el aumento de las temperaturas y los cambios en la duración de las estaciones del año, de manera conjunta, han prolongado la duración de la época reproductiva, reducido la mortalidad por el invierno y adelantado la fecha cuando emergen los insectos en la primavera \cite{Kiritani_2013}. Por tanto, el número de generaciones anuales de insectos se ha visto favorecido \cite{Harvey_2022}, generando poblaciones cada vez más grandes.\\
	
	Del total de especies de insectos, solo el 3\% se consideran plagas de interés agrícola debido a su capacidad de colonización y rápida reproducción \cite{Garcia-Lara_2016}. No obstante, se proyecta que conforme aumenten las temperaturas también crezca el número de especies con relevancia agrícola, lo que implica mayores daños para la industria \cite{Deutsch_2018}.
	}
	
	\item[]\textbf{Debilitamiento de la polinización\\}\\
	{
	La polinización se refiere, de manera conjunta, a todos aquellos procesos de regulación que intervienen en la recolección, transferencia y deposición de granos de polen en las flores \cite{Liss_2013}. De igual manera, se contemplan aspectos como la frecuencia de tales eventos, los agentes que los llevan a cabo, los métodos a través de los cuales ocurren y la cobertura geográfica \cite{Carvalheiro_2010, Westerkamp_2000}.\\
	
	Los agentes polinizadores, en la naturaleza, incluyen a los insectos, las aves, mamíferos, el viento y el agua \cite{Amrutwad_2024}. La polinización zoógama, es decir, aquella en donde interviene la presencia e interacción de algún animal, destaca del resto pues se estima que más del 75\% de las flores en el mundo dependen de ella para mantener las poblaciones de plantas existentes \cite{Ollerton_2011}.\\
	
	La polinización anemófila, aquella que se lleva a cabo a través del viento, es considerada como la segunda fuente más importante dado que su aportación en la preservación de las plantas supera el 10\% \cite{Friedman_2009}. Más aún, respecto a cultivos para consumo, las anteriores formas se consideran como las únicas alternativas dado que la polinización hidrófila está presente en un reducido número de especies de plantas exclusivamente, no abundantes y sin interés en términos alimentarios para los humanos \cite{Clemente_2023}.\\
	
	Respecto a los polinizadores animales, el panorama es similar a lo discutido previamente referente a las plagas por insectos en tanto a la migración. En general, se ha observado la pérdida de hábitats de polinizadores \cite{Memmott_2007}, la cual ha implicado la migración y la redistribución geográfica de numerosas especies \cite{Kumar_2024, Sherwin_2012, Rafferty_2017}, derivado por la elevación en temperaturas, la disponibilidad de recursos y cambios en las relaciones bióticas.\\
	
	Adicionalmente, respecto a los insectos se ha reportado una reducción en su longevidad a consecuencia de alteraciones químicas en el néctar de las flores y en las concentraciones proteicas del polen \cite{Hoover_2012, Ziska_2016}. Tales cambios se desencadenan en ambientes con altas concentraciones de CO$_2$ y pocas precipitaciones \cite{Rafferty_2017}.\\
	
	Además, las sequías reducen el tamaño floral y la intensidad de colores \cite{Burkle_2016, Rafferty_2017}, volviéndolas menos visibles o llamativas para los polinizadores. De igual forma, los cultivos pueden verse afectados en aspectos genéticos puesto que la dispersión de semillas por parte de las aves se reduce \cite{Anderson_2016}.\\
	
	Respecto a los mamíferos, estos son responsables de polinizar plantas de difícil acceso para insectos y aves \cite{Carthew_1997}. De ellos destacan principalmente los murciélagos quienes llevan a cabo la polinización de plantas cuyas flores están disponibles en periodos sin luz solar, como el agave \cite{Trejo-Salazar_2016}.\\
	
	Por otra parte, la polinización anemófila es afectada directamente por cambios en la dirección, velocidad, humedad y temperatura del viento \cite{Cresswell_2010}. En consecuencia, tales condiciones modifican la humedad en el polen, reduciendo su viabilidad cuando está falto de humedad y dificultando su motilidad al estar húmedo en exceso \cite{Fonseca_2005}.\\
	
	Adicionalmente, si el viento transporta al polen por más de dos horas, este puede dejarlo inviable por el secado o por el aislamiento causado por haber sido depositado fuera de un área de cultivo \cite{Luna_2001}. El caso de la polinización hidrófila resulta similar, dado que las corrientes pueden reducir la fertilidad y depositarlo en sitios sin relevancia reproductiva \cite{Ackerman_2000}.\\
	
	La polinización tiene un papel crucial en la supervivencia, prevalencia y variabilidad genética en las plantas \cite{Kearns_1998}. Más aún, para fines de consumo, es también fundamental para la cosecha de productos y semillas de buena calidad, tanto en sabor como en tamaño \cite{Amrutwad_2024}, lo cual repercute más notablemente en los cultivos perennes por el tiempo que se mantienen en el campo \cite{Rafferty_2017}.\\ 
	
}
	
	\item[]\textbf{Salud de los cultivos\\}\\
	Similar a lo que ocurre en los seres humanos, los cultivos se encuentran expuestos a diversos organismos patógenos que pueden comprometer su salud \cite{Savary_2020}. Estas enfermedades pueden adquirirse tanto a través de los órganos subterráneos \cite{Katan_2017} como en los órganos que yacen al aire libre \cite{Aylor_2003}, y de acuerdo con su patrón epidémico y sus consecuencias se clasifican en crónicas, agudas o emergentes \cite{Savary_2017}.\\
	
	Las enfermedades crónicas son aquellas que ocurren regularmente en cada temporada y afectan amplias extensiones de cultivo \cite{Zadoks_1979}. En contraste, las enfermedades agudas aparecen de manera irregular tanto en el tiempo como en el espacio \cite{Savary_2020, Zadoks_1979}. Finalmente, las enfermedades emergentes son aquellas que se encuentran en proceso de expansión, ya sea en su distribución geográfica o en la gama de especies que logran infectar \cite{Anderson_2004}.\\
	
	Los agentes responsables de fitoenfermedades incluyen a los hongos, virus, bacterias y nematodos  \cite{Savary_2017B}. El tamaño y localización de sus poblaciones se ha modificado en consecuencia del cambio climático generando una tendencia a que más enfermedades sean consideradas como emergentes, el reporte de nuevas enfermedades y mayor virulencia \cite{Lahlali_2024}.\\
	
	Lo anterior no solo repercute directamente en la agricultura, sino que pone en riesgo la integridad ecológica del ecosistema, en general \cite{Lahlali_2024}. Por tanto, la estabilidad de los microbiomas del suelo y la superficie se ve comprometida rápidamente, afectando su prevalencia a largo plazo \cite{Jansson_2020}, al reducir la diversidad microbiana, alterar la descomposición de materia orgánica y la retención de nutrientes \cite{Iqbal_2025}.\\	
	
	De esta manera, los daños ocasionados por patógenos, como la aceleración del secado de las hojas, la reducción de la fotosíntesis o la pérdida de firmeza de los tejidos vegetales, se presentan con mayor frecuencia \cite{Savary_2020}. Además, la simultánea disminución de los microorganismos benéficos, que ciclan nutrientes y mitigan patógenos, favorece un desequilibrio agrícola que pone en riesgo la salud de los cultivos \cite{Jansson_2020}.
	
	\item[]\textbf{Competencia entre plantas\\}\\
	
	En las secciones previas hemos explorado las relaciones biológicas entre los cultivos y diversos organismos; sin embargo, el cambio climático también ha transformado las interacciones que ocurren entre las propias especies vegetales \cite{Guo_2025}. En esta sección nos enfocaremos en analizar las dinámicas de competencia que los cultivos establecen con la maleza, las plantas parasitarias, los cultivos introducidos y aquellos generados mediante organismos genéticamente modificados.\\
	
	En términos generales, la maleza agrupa a todas aquellas plantas en un sistema agrícola que crecen de manera no deseada dado que poseen mecanismos de dispersión adaptables y eficientes \cite{Rana_2016}. A nivel nutricional, la maleza se comporta de manera similar a los cultivos; no obstante, por cuestiones genéticas, su agresividad y capacidad de expansión biogeográfica pueden verse favorecidas por el alza en las concentraciones de CO$_2$, las altas temperaturas y el estrés hídrico \cite{Upasani_2018}.\\
	
	Lo anterior implica que la maleza, como competencia para los cultivos, posee habilidades de adaptación y una fuerte capacidad de capturar los recursos disponibles \cite{Rana_2016}. En contraste, respecto a las plantas parasitarias también se ha reportado una ampliación tanto geográfica como en la gama de especies hospederas \cite{Zamora_2019}.\\
	
	En el caso de las plantas parásitas, estas no compiten directamente por los recursos, sino que los extraen de manera indirecta de la planta hospedera \cite{Watson_2022}. Lo anterior provoca un reajuste en la distribución de la energía reservada por la planta, lo que conlleva a un aumento de la transpiración en el hospedero, frecuentemente asociado a un enfriamiento de sus tejidos \cite{Pincebourde_2019}.\\
	
	Este fenómeno reduce la producción de flores y frutos \cite{Watson_2022}, pero puede implicar consecuencias fatales. En zonas donde, a consecuencia del cambio climático, los recursos hídricos han incrementado, este efecto genera condiciones de alta humedad que favorecen el desarrollo de co-infecciones por patógenos \cite{Velasquez_2018}. Mientras que en zonas con escasez de agua, el incremento en la transpiración puede resultar mortal tanto para la planta parásita como para el hospedero \cite{Watson_2022}.\\
	
	Por tanto, la presencia tanto de maleza como de plantas parasitarias es indeseable, desde un punto de vista económico y productivo. Desde esta perspectiva, también destaca la introducción de cultivos no nativos y aquellos generados mediante organismos genéticamente modificados, los cuales se abordan a continuación.\\
	
	En un sistema agrícola, los cultivos no nativos pueden ser introducidos de manera natural, al aprovechar oportunidades de colonización; de manera artificial, por decisión humana; o mediante una combinación de ambas estrategias \cite{Webber_2012}. En cualquiera de los casos, esto afecta directamente a la biodiversidad local y la sinergia del ecosistema \cite{Sorte_2013}.\\
	
	La introducción de especies ocurre principalmente por los cambios en el ambiente \cite{Byers_2002}. En cuanto a la introducción natural, esta ocurre debido a que las especies han hallado condiciones más favorables para su desarrollo \cite{Sorte_2013}. Mientras que la introducción artificial se lleva a cabo, principalmente, por cuestiones productivas y su rentabilidad comercial \cite{Hodge_1956}.\\
	
	En tierras de cultivo, mayoritariamente, la introducción de especies se realiza de manera artificial \cite{Steen_2024}; mientras que en cultivos acuáticos, la introducción de especies vegetales se realiza principalmente de manera natural \cite{Sorte_2013}. No obstante, la sustitución de cultivos nativos por especies no nativas ha ido en aumento; e.g., en Europa se estima que el 40\% de las especies nativas serán sustituidas para el año 2080 \cite{Thuiller_2005}.\\
	
	A diferencia de los cultivos no nativos, cuya introducción puede ser natural o artificial, los cultivos genéticamente modificados dependen exclusivamente de la intervención humana. En este caso, las compañías fabricantes diseñan y editan las semillas con el fin de obtener cultivos resistentes a condiciones ambientales adversas, plagas y enfermedades \cite{Verma_2011} a la par de asegurar una alta producción de calidad nutrimental y comercial \cite{Cheng_2024}.\\
	
	
	Por otra parte, al consumir los productos resultantes la edición genética, potencialmente pueden presentarse nuevas reacciones alérgicas impredecibles y alteraciones en los sistemas digestivo e inmunológico \cite{Verma_2011}. Adicionalmente, este tipo de cultivos suele incorporar tecnologías de restricción de uso genético, coloquialmente conocidas como \textit{genes suicidas}, que hacen infértiles a las semillas que se obtienen, por lo que estos cultivos tienen un único ciclo productivo \cite{Daniell_2002}.\\
	
	
	Lo anterior se realiza como una medida de bioseguridad; no obstante tiene repercusiones económicas adversas para los productores \cite{VanAcker_2007}. De igual forma, al adoptar este tipo de cultivos, las especies nativas y, en especial, aquellas endémicas se ven altamente amenazadas corriendo, incluso, el riesgo de la extinción \cite{Kumar_2020}.\\
	
	
	En gran medida, la desventaja que presentan los cultivos frente a otras especies vegetales puede atribuirse al refinamiento genético al que han sido sometidos durante el proceso de domesticación \cite{Moreira_2018}. Este proceso busca obtener características productivas deseables, pero ha reducido significativamente la agresividad y capacidad competitiva de los cultivos frente a otras plantas \cite{DaSilva_2007, Concenco_2012}.
	
\end{itemize}


\begin{table}[ht]
	\centering
	\caption{Consecuencias de origen biótico del cambio climático sobre los cultivos}
	\begin{tabularx}{\textwidth}{|l|X|}
		\hline
		\textbf{Consecuencia sobre el cultivo} & \textbf{Descripción resumida / Factores causales} \\ \hline
		Aumento de plagas y patógenos & Incremento en incidencia de insectos, hongos, bacterias y virus favorecidos por cambios en temperatura y humedad. \\ \hline
		Competencia con malezas y especies invasoras & Condiciones climáticas modificadas favorecen la expansión de malezas y especies introducidas que desplazan a cultivos. \\ \hline
		Interacción con plantas parasitarias y transgénicas & Plantas parásitas (ej. \textit{Striga}) aumentan bajo estrés hídrico; flujo génico de transgénicos altera dinámica competitiva. \\ \hline
		Alteraciones en microorganismos benéficos & Reducción de hongos micorrízicos y bacterias fijadoras de nitrógeno por estrés ambiental disminuye la fertilidad natural del suelo. \\ \hline
		Deficiencia en la polinización & Cambios en temperatura, viento o lluvias afectan la actividad de insectos polinizadores y la viabilidad del polen. \\ \hline
		Salud general de los cultivos & Incremento de daños por interacción múltiple de plagas, patógenos y deficiencias fisiológicas inducidas por el clima. \\ \hline
	\end{tabularx}
\end{table}


\subsubsection{Factores socioeconómicos}
\begin{itemize}[leftmargin=0cm, itemsep=0.5 cm]
\item[]\textbf{Superficies cultivables\\}
\item[]\textbf{Escasez laboral\\}
\item[]\textbf{Volatilidad productiva\\}
\item[]\textbf{Toxicidad por agroquímicos\\}
\end{itemize}

\begin{table}[ht]
	\centering
	\caption{Consecuencias de origen socioeconómico del cambio climático sobre los cultivos}
	\begin{tabularx}{\textwidth}{|l|X|}
		\hline
		\textbf{Consecuencia sobre el cultivo} & \textbf{Descripción resumida / Factores causales} \\ \hline
		Escasez laboral & Migración, envejecimiento del campo y condiciones climáticas extremas reducen la disponibilidad de mano de obra agrícola. \\ \hline
		Incremento en costos de producción & Necesidad de riego, fertilización adicional, control de plagas y tecnologías de adaptación eleva los costos. \\ \hline
		Reducción de tierras cultivables & Expansión urbana, desertificación y pérdida de suelos fértiles disminuyen la superficie agrícola disponible. \\ \hline
		Cambio en mercados y precios agrícolas & Variabilidad en la producción genera inestabilidad en precios y acceso desigual a alimentos. \\ \hline
		Inseguridad alimentaria & Pérdida de rendimientos y volatilidad económica afectan la disponibilidad y acceso a alimentos básicos. \\ \hline
		Adaptación tecnológica desigual & Agricultores con menos recursos tienen menor acceso a semillas resistentes, sistemas de riego y tecnologías de precisión. \\ \hline
	\end{tabularx}
\end{table}
