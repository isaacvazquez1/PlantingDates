A gran escala, la producción agrícola responde a una amplia variedad de elementos que, según su procedencia, permanencia, frecuencia e intensidad, pueden modificar su desarrollo de manera neutra, favorable o adversa. En esta sección se abordarán aquellos aspectos cuya intervención incide en la agenda de actividades productivas, como la preparación del terreno, las fechas de siembra, el trasplante y la cosecha, planificadas en función de las características biológicas y agroambientales de los cultivos a establecer.\\

Es decir, nos centraremos en analizar las cuestiones que, de manera directa o indirecta, influyen en la sincronía entre el calendario agrícola y la fenología de los cultivos, una relación que, en sus inicios, se sustentó en observaciones basadas en el conocimiento ancestral de los pueblos originarios y que, en la actualidad, se ha visto fortalecida por el desarrollo tecnológico y la investigación interdisciplinaria.