El proyecto aquí propuesto se ubica dentro de la línea de generación y aplicación del conocimiento de analítica de grandes cúmulos de información y pretende abordar la pregunta de investigación referente a cómo pueden utilizarse técnicas de aprendizaje automático, considerando información agroclimática, para predecir la evolución del proceso productivo del aguacate seccionándolo en las distintas etapas que lo componen. Esto, bajo la hipótesis de que, en efecto, es viable generar tales pronósticos.\\

De esta manera, emplearemos modelos de aprendizaje automático; por ejemplo, algoritmos de regresión, support vector machine, árboles de decisión o random forest, los cuales serán alimentados con datos agroclimáticos históricos para pronosticar las fechas más apropiadas para realizar el trasplante al suelo, la producción de flores y la producción de aguacates. Los anterior, tomando en consideración los objetivos que presentamos a continuación.\\

Objetivos generales.

\begin{enumerate}
    \item Proponer un modelo de predicción, basado en aprendizaje automático, para determinar las fechas óptimas de trasplante de árboles de aguacate, la producción de flores y producción de aguacates en Michoacán, considerando variables climáticas y edáficas.
    \item Contribuir en la toma de decisiones dentro del proceso productivo del aguacate analizando datos históricos y tendencias meteorológicas, en el estado de Michoacán, para reducir el impacto del cambio climático dentro de este proceso.
\end{enumerate}

Objetivos particulares.

\begin{enumerate}
    \item Realizar una revisión bibliográfica exhaustiva sobre modelos predictivos basados en técnicas de aprendizaje automático orientados hacia la industria agrícola, enfocándonos en aquellos trabajos que se han interesado en cultivos con características similares a las del aguacate.
    \item Definir y depurar las bases de datos a utilizar, considerando registros históricos de clima, suelo y producción de aguacates en Michoacán.
    \item Estudiar la participación de las variables disponibles y determinar cuáles tienen mayor peso y relevancia en cada una de las etapas del proceso productivo, desde el la siembra hasta la cosecha
    \item Explorar y comparar diferentes técnicas de aprendizaje automático para determinar cuáles presentan mejor desempeño y exactitud como modelo predictivo.
    \item Entrenar y validar el modelo predictivo desarrollado mediante pruebas con datos recientes para evaluar su precisión.
    \item Publicar los resultados obtenidos a través de la elaboración de la tesis, la redacción de artículos científicos y presentaciones en congresos.
\end{enumerate}