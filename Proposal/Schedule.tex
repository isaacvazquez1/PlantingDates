Para asegurar el desarrollo y la culminación exitosa del proyecto, se propone el siguiente cronograma, el cual se ha desglosado semestralmente. Lo anterior, para tomarlo como referencia a fin de garantizar un avance progresivo al apegarnos lo más posible a él y cubrir tanto los aspectos referentes al proyecto de investigación como los aspectos administrativos.\\

\begin{longtable}{|c|p{0.68\textwidth}|} 
    \hline
    \textbf{Semestre} & \multicolumn{1}{c|}{\textbf{Actividades}} \\ 
    \hline
    \endfirsthead

    \hline
    \textbf{Semestre} & \multicolumn{1}{c|}{\textbf{Actividades}} \\ 
    \hline
    \endhead
    
    \centering Agosto -- Diciembre 2025 & 
    
    \begin{list}{\textbullet}{%
    \setlength{\leftmargin}{0.2cm}%
    \addtolength{\topsep}{-1.1cm}%
    \addtolength{\labelsep}{-2mm}%
    \addtolength{\itemsep}{-2mm}%
    \setlength{\itemindent}{0cm}}
        \item Profundizar la revisión bibliográfica.
        \item Definir y depurar el conjunto de datos con el que se trabajará a lo largo de proyecto.
        \item Explorar técnicas de aprendizaje automático aplicadas a cultivos, tanto para aguacates como a algunos otros cultivos de características similares.
        \item Elaborar una primera versión del protocolo de investigación.
    \end{list} \\ \hline
    \centering Enero -- Julio 2026      & \begin{list}{\textbullet}{%
    \setlength{\leftmargin}{0.2cm}%
    \addtolength{\topsep}{-1.1cm}%
    \addtolength{\labelsep}{-2mm}%
    \addtolength{\itemsep}{-2mm}%
    \setlength{\itemindent}{0cm}}
        \item Proponer, diseñar y revisar la metodología del modelo de predicción.
        \item Implementar un análisis exploratorio con datos preliminares.
        \item Presentar la versión final del protocolo de investigación ante el comité sinodal.
        \item Comenzar con la escritura de la tesis.
    \end{list} \\ \hline
    \centering Agosto -- Diciembre 2026 & \begin{list}{\textbullet}{%
    \setlength{\leftmargin}{0.2cm}%
    \addtolength{\topsep}{-1.1cm}%
    \addtolength{\labelsep}{-2mm}%
    \addtolength{\itemsep}{-2mm}%
    \setlength{\itemindent}{0cm}}
        \item Entrenar modelos preliminares de predicción con diferentes técnicas.
        \item Evaluar el desempeño de los modelos utilizando métricas estándar.
        \item  Comenzar con la escritura del primer artículo de investigación.
        \item Presentar avances en un congreso nacional; por ejemplo, el Congreso Nacional de la Sociedad Matemática Mexicana, Congreso Nacional de Agricultura Sostenible u otros.
    \end{list} \\ \hline
    \centering Enero -- Julio 2027      & \begin{list}{\textbullet}{%
    \setlength{\leftmargin}{0.2cm}%
    \addtolength{\topsep}{-1.1cm}%
    \addtolength{\labelsep}{-2mm}%
    \addtolength{\itemsep}{-2mm}%
    \setlength{\itemindent}{0cm}}
        \item Identificar el congreso internacional más adecuado para el envío del artículo.
        \item Preparar una presentación oral o póster para dicho evento
        \item Enviar el artículo que se desprenda del proyecto a un congreso internacional.
        \item Fortalecer el proyecto a partir de buscar colaboraciones potenciales nacionales e internacionales.
    \end{list} \\ \hline
    \centering Agosto -- Diciembre 2027 & \begin{list}{\textbullet}{%
    \setlength{\leftmargin}{0.2cm}%
    \addtolength{\topsep}{-1.1cm}%
    \addtolength{\labelsep}{-2mm}%
    \addtolength{\itemsep}{-2mm}%
    \setlength{\itemindent}{0cm}}
        \item  Realizar actividades de retribución social para cumplir con los requisitos de la SECIHTI.
        \item Consolidar los primeros capítulos de la tesis.
        \item Definir las revistas de investigación, indexadas en el JCR y con un buen factor de impacto, en las cuales nuestro proyecto encajaría y destacaría.
    \end{list} \\ \hline
    \centering Enero -- Julio 2028      & \begin{list}{\textbullet}{%
    \setlength{\leftmargin}{0.2cm}%
    \addtolength{\topsep}{-1.1cm}%
    \addtolength{\labelsep}{-2mm}%
    \addtolength{\itemsep}{-2mm}%
    \setlength{\itemindent}{0cm}}
        \item Refinar y evaluar las metodologías para hacer ajustes considerando los resultados obtenidos y los alcances del proyecto. 
        \item Revisar y confirmar que el artículo a enviar cumple con el proceso editorial de la revista.
        \item Enviar el primer artículo de investigación que se desprenda del proyecto a una revista indexada en el JCR.
    \end{list} \\ \hline
    \centering Agosto -- Diciembre 2028 & \begin{list}{\textbullet}{%
    \setlength{\leftmargin}{0.2cm}%
    \addtolength{\topsep}{-1.1cm}%
    \addtolength{\labelsep}{-2mm}%
    \addtolength{\itemsep}{-2mm}%
    \setlength{\itemindent}{0cm}}
        \item Revisar comentarios de los revisores del artículo enviado para llevar a cabo las correcciones pertinentes en las rondas de revisiones que sean solicitadas. 
        \item Integrar las correcciones discutidas durante la revisión del artículo a los capítulos centrales de la tesis.
        \item Revisar y releer los avances de la tesis para afinar detalles, evaluar el cumplimiento de los objetivos planteados y dar paso a la conclusión del trabajo.
    \end{list} \\ \hline
    \centering Enero -- Julio 2029      & \begin{list}{\textbullet}{%
    \setlength{\leftmargin}{0.2cm}%
    \addtolength{\topsep}{-1.1cm}%
    \addtolength{\labelsep}{-2mm}%
    \addtolength{\itemsep}{-2mm}%
    \setlength{\itemindent}{0cm}}
        \item Obtener la aceptación del artículo enviado a una revista indexada en el JCR.
        \item Culminar la escritura de la tesis con el visto bueno por parte de la directora de tesis.
        \item Enviar la tesis al comité sinodal para su revisión, corrección y posterior aprobación.
        \item Solicitar la autorización de impresión de la tesis por parte de la biblioteca.
        \item Aprobar el examen cerrado ante el comité sinodal para, posteriormente, realizar el examen de defensa de tesis para la obtención del grado académico.
    \end{list} \\ \hline
\end{longtable}

 